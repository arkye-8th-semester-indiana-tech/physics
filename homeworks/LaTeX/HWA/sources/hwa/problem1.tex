\section{Problem (1)}
	A good baseball pitcher can throw a baseball toward home plate at $95 \ mi/h$ with a spin of $1300 \ rev/min$. How many revolutions does the baseball make on its way to home plate? For simplicity, assume that the $60 \ ft$ path is a straight line.

	\textbf{R:}

	\begin{align}
		v = \ &95 \ mi/h
		\times \left( \frac{5280 \ ft}{1 \ mi} \right)
		\times \left( \frac{1 \ h}{3600 \ s} \right)
		& \notag \\
		= \ &139.\overline{3} \ ft/s& \notag \\
		\omega = \ &1300 \ rev/min
		\times \left( \frac{1 \ min}{60 \ s} \right)
		& \notag \\
		= \ &21.\overline{6} \ rev/s& \notag \\
		t = \ &\frac{x}{v} = \frac{60 \ ft}{139.\overline{3} \ ft/s}& \notag \\
		= \ &0.431 \ s& \notag \\
		n_{rev} = \ &(21.\overline{6} \ rev/s)(0.431 \ s) = 9.33 \ rev&
	\end{align}
