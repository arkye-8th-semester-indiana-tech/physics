\section{Problem (6)}
	A collie drags its bed box across a floor by applying a horizontal force of $3.1 \ lb$. The kinetic frictional force acting on the box has magnitude $1.8 \ lb$. As the box is dragged at constant speed through $4.2 \ ft$ along the way:

	\subsection{Question (a)}

		What is the work done by the collie's applied force?

		\textbf{R:}

		\begin{align}
			W_{\text{collie}} = \ &\vec{F}_{\text{collie}} \ \Delta \vec{r}& \notag \\
			= \ &(3.1 \ lb)(4.2 \ ft)& \notag \\
			= \ &13.02 \ ft \times lb& \notag
		\end{align}

	\subsection{Question (b)}

		What is the work done by the friction force?

		\textbf{R:}

		\begin{align}
			W_{F_{s}} = \ &\vec{F}_{s} \Delta \vec{r}& \notag \\
			= \ &-(1.8 \ lb)(4.2 \ ft)& \notag \\
			= \ &-7.56 \ ft \times lb& \notag
		\end{align}
