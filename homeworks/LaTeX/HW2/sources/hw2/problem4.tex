\section{Problem (4)}
	On a dry road, a car with good tires may be able to brake with a constant deceleration of $5.6 \ m/s^{2}$.

	\subsection{Question (a)}
		How long does such a car, initially travelling at $29 \ m/s$, take to stop?

		\textbf{R:} \newline
		\begin{align}
			v_{0} = \ &29 \ m/s& \notag \\
			v = \ &0 \ m/s& \notag \\
			a = \ &- 5.6 \ m/s^{2}& \notag \\
			v = \ &v_{0} + at& \notag \\
			0 \ m/s = \ &(29 \ m/s) + \left( -5.6 \ m/s^{2} \right) t& \notag \\
			t = \ &\frac{29 \ m/s}{5.6 \ m/s^{2}} = 5.2 \ s&
		\end{align}
	\subsection{Question (b)}
		How far does it travel in this time?

		\textbf{R:} \newline
		\begin{align}
			x_{0} = \ &0 \ m& \notag \\
			x = \ &x_{0} + v_{0}t + \frac{1}{2}at^{2}& \notag \\
			= \ &(0 \ m) + (29 \ m/s)(5.2 \ s) + \frac{1}{2} \left( -5.6 \ m/s^{2} \right)(5.2 \ s)^{2}& \notag \\
			= \ &(150.8 \ m) - (-75.7 \ m) = 75.1 \ m&
		\end{align}
