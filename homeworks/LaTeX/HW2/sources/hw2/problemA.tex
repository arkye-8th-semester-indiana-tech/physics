\section{Problem (10)}
	At the instant the traffic light turns green, an automobile starts with a constant acceleration of $6.7 \ ft/s^{2}$. At the same instant a truck, traveling with a constant speed of $32 \ ft/s$, overtakes and passes the automobile.

	\subsection{Question (a)}
		How far beyond the traffic signal will the automobile overtake the truck?

		\textbf{R:} \newline
		\begin{align}
			&\text{Automobile}& &\text{Truck}& \notag \\
			&x_{A_{0}} = 0 \ ft& &x_{T_{0}} = 0 \ ft& \notag \\
			&x_{A} = x_{T}& &x_{T} = x_{A}& \notag \\
			&v_{A_{0}} = 0 \ ft/s& &v_{T_{0}} = 32 \ ft/s& \notag \\
			&v_{A} = \text{?} \ ft/s& &v_{T} = v_{T_{0}} = 32 \ ft/s& \notag \\
			&a_{A} = 6.7 \ ft/s^{2}& &a_{T} = 0 \ ft/s^{2}& \notag
		\end{align}

		\begin{align}
			x = \ &x_{0} + v_{0}t + \frac{1}{2}at^{2}& \notag \\
			x_{A} = \ &(0 \ ft) + (0 \ ft/s)t + \frac{1}{2} \left( 6.7 \ ft/s^{2} \right) t^{2}& \notag \\
			= \ & \left( 3.35 \ ft/s^{2} \right) t^{2}& \notag \\
			x_{T} = \ &(0 \ ft) + (32 \ ft/s)t + \frac{1}{2} \left( 0 \ ft/s^{2} \right) t^{2}& \notag \\
			x_{T} = \ &(32 \ ft/s)t& \notag
		\end{align}

		\begin{align}
			\left( 3.35 \ ft/s^{2} \right) t^{2} = \ &(32 \ ft/s)t& \notag \\
			t = \ &\frac{32 \ ft/s}{3.35 \ ft/s^{2}} = 9.552 \ s& \notag \\
			x_{T} = \ &(32 \ ft/s)(9.552 \ s) = 305.7 \ ft& \\
			x_{A} = \ & \left( 3.35 \ ft/s^{2} \right) (9.552 \ s)^{2}& \notag \\
			= \ & \left( 3.35 \ ft/s^{2} \right) (91.241 \ s^{2}) = 305.7 \ ft&
		\end{align}
	\subsection{Question (b)}
		How fast will the car be traveling at that instant?

		\textbf{R:} \newline
		\begin{align}
			v = \ &v_{0} + at& \notag \\
			v_{A} = \ &(0 \ ft/s) + (6.7 \ ft/s^{2})(9.552 \ s) = 64 \ ft/s& \notag \\
		\end{align}
