\section{Problem (7)}
	Two disks are mounted (like a merry-go-round) on low friction bearings on the same axle and can be brought together so that they couple and rotate as one unit. The first disk, with rotational inertia $4.8 \ sl \times ft^{2}$ about its central axis, is set spinning counterclockwise (which may be taken as the positive direction) at $420 \ rev/min$. The second disk, with rotational inertia $3.6 \ sl \times ft^{2}$ about its central axis, is set spinning counterclockwise at $640 \ rev/min$. They then couple together.

	\subsection{Question (a)}

		What is the angular speed (rev/min) after coupling?

		\textbf{R:}

		\begin{align}
			x
		\end{align}

	\subsection{Question (b)}

		If instead the second disk is set spinning clockwise at $640 \ rev/min$, what is their angular velocity (in rev/min, using the correct sign for direction) after they couple together?

		\textbf{R:}

		\begin{align}
			x
		\end{align}
