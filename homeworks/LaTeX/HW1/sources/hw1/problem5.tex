\section{Problem (5)}
	For about $10$ years after the French Revolution, the French government attempted to base measures of time on multiples of ten: One week consisted of $10$ days, one day consisted of $10$ hours, one hour consisted of $100$ minutes, and one minute consisted of $100$ seconds. For the questions below, assume that the definition of a `` day '' remains the same. Note that a `` ratio '' is simply a single decimal value.

	\subsection{Question (a)}
		What are the ratio of the French decimal week to the standard week? \newline
		\textbf{R:} \newline
		Since a day in both standards are the same, the ratio can be discovered as follow:
		\begin{align}
			1 \ \text{French week} = \ & 10 \ \text{days} & \\
			1 \ \text{standard week} = \ & 7 \ \text{days} & \\
			\frac{10 \ \text{days}}{7 \ \text{days}}
			\approx \ &1.43&
		\end{align}

	\subsection{Question (b)}
		What are the ratio of the French decimal second to the standard second?
		\newline
		\textbf{R:} \newline
		Since a day in both standards are the same, the ratio can be discovered as follow:
		\begin{align}
			1 \ \text{day} = \ & 10 \ \text{French hours} & \notag \\
			= \ & 10^{3} \ \text{French minutes} & \notag \\
			= \ & 10^{5} \ \text{French seconds} & \\
			1 \ \text{day} = \ & 8.64 \times 10^{4} \ s & \\
			\frac{10^{5} \ \text{French seconds}}{8.64 \times 10^{4} \ s}
			\approx \ &1.16&
		\end{align}
