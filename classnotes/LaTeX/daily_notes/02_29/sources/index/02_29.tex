\section{Friction}
	Friction $\to$ Surfaces in contact.
	Two Types:
	\begin{enumerate}
		\item{\textbf{Static}: Surfaces \textbf{not} slipping;}
		\item{\textbf{Kinetic}: Surfaces are slipping;}
	\end{enumerate}
	Let
	\begin{align}
	\end{align}
		f_{s} = \ &\text{force of static friction}& \notag
	$f_{s}$ automatically adjusts to be exactly as big as it to be to prevent slippage, as to some maximum value $f_{s,max}$

	\begin{align}
		f_{s} \leq \ &f_{s,max}& \notag
	\end{align}

	$f_{s,max}$ depends on:
	\begin{enumerate}
		\item{Nature of the surfaces. See Table 5.2 (p. 132)

		$\mu_{s} = \text{coefficient of static friction}$}
		\begin{align}
			&\text{Surfaces}& &\mu_{s}& &\mu_{k} \notag \\
			&\text{Steel-steel}& &0.74& &0.57& \notag \\
			&\text{Ruble + concrete}& &1.0& &0.8& \notag \\
			&\text{Ice + Ice}& &0.1& &0.03& \notag
		\end{align}}
		\item{Normal force $n$ between surfaces
		\begin{align}
			f_{s,max} = \ &\mu_{s} \times n& \notag
		\end{align}
		Let
		\begin{align}
			\mu_{k} = \ &\text{coefficient of kinetic friction}& \notag \\
			\mu_{k} < \ &\mu_{s}& \notag
		\end{align}
		Let
		\begin{align}
			f_{k} = \ &\text{force of kinetic friction}& \notag \\
			f_{k} = \ &\mu_{k} \times n& \notag
		\end{align}}
	\end{enumerate}

	% Example 10
	% Extra Credit: The Angle that fs = fs,max

	\begin{align}
		f_{s} = \ &P_{x} = P \cos \theta & \notag \\
		= \ &(300 \ lb) \cos \theta & \notag \\
		f_{s_max} = \ &\mu_{s} \times n& \notag \\
		= \ &(0.50)(F_{g} - P_{y})& \notag \\
		= \ &(0.50)[(644 \ lb) - (P \sin \theta)]& \notag \\
		= \ &322 \ lb - \frac{1}{2}[(300 \ lb) \sin \theta]& \notag \\
		= \ &322 \ lb - [(150 \ lb)\sin \theta]& \notag
	\end{align}

	\begin{align}
		f_{s} = \ &f_{s_max}& \notag \\
		(300 \ lb) \cos \theta = \ &322 \ lb - [(150 \ lb)\sin \theta]& \notag \\
		(300 \ lb) \cos \theta + (150 \ lb)\sin \theta = \ &322 \ lb& \notag \\
		(150 \ lb)[(2 \cos \theta) + \sin \theta] = \ &322 \ lb& \notag \\
		(2 \cos \theta) + \sin \theta = \ &\frac{161}{75}& \notag \\
		2 + \tan \theta = \ &\frac{161}{75} \sec \theta& \notag \\
		\tan \theta = \ &(\frac{161}{75} \sec \theta) - 2& \notag \\
		\theta = \ &\tan^{-1} \left[ \left(\frac{161}{75} \sec \theta \right) - 2 \right]& \notag \\
		\theta = 42.821
		\theta = 10.308636
	\end{align}
