\section{Power}

	\begin{align}
		P = power
		\bar{P} = average power
		\bar{P} = \frac{\Delta W}{\Delta t}
	\end{align}

	\begin{align}
		P = instantaneous power
		= \frac{dW}{dt} = rate of which work is done
	\end{align}

	\begin{align}
		1 J/s = 1 W = 1 watt
	\end{align}

	\begin{align}
		1 J = 1 Nm
		1 W = 1 J/s = 1 Nm/s = 1 kgm^{2}/s^{3}
	\end{align}

	\begin{align}
		1 ftlb/s = 1 slft^{2}/s^{3}
		1 hp = 1 horse power = 550 ftlb/s = 745.7 W
	\end{align}

	\begin{align}
		P = \frac{dW}{dt} = \frac{\vec{F}\Delta\vec{r}}{dt}
		= \vec{F} \frac{d\vec{r}}{dt} = \vec{F}\vec{v}
	\end{align}

	Example:

	A $500 \ kg$ elevator is lifted steadly at $6 \ m/s$. What power does the power put on?

	FBD:
	down -> fg = mg
	up -> n, v
	a = 0
	-----

	\begin{align}
		\sum F_y = ma_y
		+n - mg = m(0)
		n = mg
		= (500 kg)(9.8 m/s^{2})
		= 4900 N
		P = \vec{F}\vec{v}
		= nv \cos \theta
		= (4900 N)(6 m/s) \cos 0
		= 29400 J/s = 29400 W
		= 29400 W \times \frac{1 hp}{745.7 W}
		= 39.4 hp
	\end{align}

	More generally

	\begin{align}
		power = \frac{energy transformed}{time}
	\end{align}

	A $100 W$ light bulb converts $100 J$ of electrical energy into $100 J$ of light and heat every second.

	\begin{align}
		100 W = 100 J/s
	\end{align}

	What is a kilowatt-hour?

	\begin{align}
		1 kWh = (1000 J/s)(3600 s)
		= 3,600,00 J (Energy)
		= 3.6 \times 10^{6} J
		= 3.6 MJ
	\end{align}

	% Example 15
