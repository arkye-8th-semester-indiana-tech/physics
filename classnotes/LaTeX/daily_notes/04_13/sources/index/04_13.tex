\section{End of Chapter 9}
	1) Translation: Motion from point to point
	2) Rotation: Rotating about an axis

	An object translates as through all of its mass were located at the point called the ``center of mass".

	\begin{align}
		M = total mass of object
		\vec{R}_{cm} = position of CM
		\vec{V}_{cm} = velocity of CM
		\vec{A}_{cm} = acceleration of CM
	\end{align}

	\begin{align}
		\vec{P}_{tot} = M\vec{V}_{cm}
		\vec{F}_{tot} = M\vec{A}_{cm}
		K = \frac{1}{2} M\vec{V}_{cm}^{2}
	\end{align}

	Suppose an object consists of $n$ point particles of masses $m_{1}, m_{2}, ... , m_{n}$ at position $\vec{r}_{1}, \vec{r}_{2}, ... , \vec{r}_{n}$.

	\begin{align}
		M = m_{1} + m_{2} + ... + m_{n}
		\vec{R}_{cm} = \frac{m_{1}\vec{r}_{1} + m_{2}\vec{r}_{2} + ... + m_{n}\vec{r}_{n}}{M}
		x:
		X_{cm} = \frac{m_{1}x_{1} + m_{2}x_{2} + ... + m_{n}x_{n}}{M}
		y:
		Y_{cm} = \frac{m_{1}y_{1} + m_{2}y_{2} + ... + m_{n}y_{n}}{M}
		z:
		Z_{cm} = \frac{m_{1}z_{1} + m_{2}z_{2} + ... + m_{n}z_{n}}{M}
	\end{align}

	Example: Lop-sided barbell
	h = 7m
	m1 = 4kg
	m2 = 6kg

	\begin{align}
		m_{1} = 4 kg
		x_{1} = 0 m
		m_{2} = 6 kg
		x_{2} = 7 m
		M = m_{1} + m_{2}
		= 4 + 6 = 10
		X_{cm} = \frac{m_{1}x_{1} + m_{2}x_{2}}{M}
		= \frac{(4 kg)(0 m) + (6 kg)(7 m)}{10 m}
		= \frac{0 + 42 kgm}{10 kg} = \frac{42 kgm}{10 kg} = 4.2 m
	\end{align}

	Example:
	m1 = 3 kg
	r1 = 0 m, 0 m
	m2 = 2 kg
	r2 = 0 m, 7 m
	m3 = 5 kg
	r3 = 6 m, 7 m

	\begin{align}
		M = 3 kg + 2 kg + 5 kg = 10 kg
		X_{cm} = 3 m
		Y_{cm} = \frac{14 kgm + 35 kgm}{10 kg} = 4.9 m
	\end{align}

	Example: Metal Plate (Uniform density and thickness)
   __
	|  |__
	|_____|

	p1 = rectangle
	p2 = square

	\begin{align}
		m_{1} = 2 x 4 = 8
		x_{1} = 1 y_{1} = 2
		m_{2} = 2 x 2 = 4
		x_{2} = 3 y_{2} = 1
		M = 12
		X_{cm} = \frac{m_1 x_1 + m_2 x_2}{M}
		= \frac{8(1) + 4(3)}{12} = 20/12 = 1.67
		Y_{cm} = \frac{m_1 y_1 + m_2 y_2}{M}
		= \frac{8(2) + 4(1)}{12} = 20/12 = 1.67
	\end{align}

	Example:
	__ __
 |  |__|
 |_____|

 \begin{align}
	 m_{1} = 4 x 4 = 16
	 x_{1} = 2 y_{1} = 2
	 m_{2} = -(2 x 2) = -4
	 x_{2} = 3 y_{2} = 3
	 M = 12
	 X_{cm} = \frac{m_1 x_1 + m_2 x_2}{M}
	 = \frac{16(2) + (-4)(3)}{12} = 20/12 = 1.67
	 Y_{cm} = \frac{m_1 y_1 + m_2 y_2}{M}
	 = \frac{16(2) + (-4)(3)}{12} = 20/12 = 1.67
 \end{align}
